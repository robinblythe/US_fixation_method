% Options for packages loaded elsewhere
\PassOptionsToPackage{unicode}{hyperref}
\PassOptionsToPackage{hyphens}{url}
%
\documentclass[
]{article}
\usepackage{amsmath,amssymb}
\usepackage{iftex}
\ifPDFTeX
  \usepackage[T1]{fontenc}
  \usepackage[utf8]{inputenc}
  \usepackage{textcomp} % provide euro and other symbols
\else % if luatex or xetex
  \usepackage{unicode-math} % this also loads fontspec
  \defaultfontfeatures{Scale=MatchLowercase}
  \defaultfontfeatures[\rmfamily]{Ligatures=TeX,Scale=1}
\fi
\usepackage{lmodern}
\ifPDFTeX\else
  % xetex/luatex font selection
\fi
% Use upquote if available, for straight quotes in verbatim environments
\IfFileExists{upquote.sty}{\usepackage{upquote}}{}
\IfFileExists{microtype.sty}{% use microtype if available
  \usepackage[]{microtype}
  \UseMicrotypeSet[protrusion]{basicmath} % disable protrusion for tt fonts
}{}
\makeatletter
\@ifundefined{KOMAClassName}{% if non-KOMA class
  \IfFileExists{parskip.sty}{%
    \usepackage{parskip}
  }{% else
    \setlength{\parindent}{0pt}
    \setlength{\parskip}{6pt plus 2pt minus 1pt}}
}{% if KOMA class
  \KOMAoptions{parskip=half}}
\makeatother
\usepackage{xcolor}
\usepackage[margin=1in]{geometry}
\usepackage{color}
\usepackage{fancyvrb}
\newcommand{\VerbBar}{|}
\newcommand{\VERB}{\Verb[commandchars=\\\{\}]}
\DefineVerbatimEnvironment{Highlighting}{Verbatim}{commandchars=\\\{\}}
% Add ',fontsize=\small' for more characters per line
\usepackage{framed}
\definecolor{shadecolor}{RGB}{248,248,248}
\newenvironment{Shaded}{\begin{snugshade}}{\end{snugshade}}
\newcommand{\AlertTok}[1]{\textcolor[rgb]{0.94,0.16,0.16}{#1}}
\newcommand{\AnnotationTok}[1]{\textcolor[rgb]{0.56,0.35,0.01}{\textbf{\textit{#1}}}}
\newcommand{\AttributeTok}[1]{\textcolor[rgb]{0.13,0.29,0.53}{#1}}
\newcommand{\BaseNTok}[1]{\textcolor[rgb]{0.00,0.00,0.81}{#1}}
\newcommand{\BuiltInTok}[1]{#1}
\newcommand{\CharTok}[1]{\textcolor[rgb]{0.31,0.60,0.02}{#1}}
\newcommand{\CommentTok}[1]{\textcolor[rgb]{0.56,0.35,0.01}{\textit{#1}}}
\newcommand{\CommentVarTok}[1]{\textcolor[rgb]{0.56,0.35,0.01}{\textbf{\textit{#1}}}}
\newcommand{\ConstantTok}[1]{\textcolor[rgb]{0.56,0.35,0.01}{#1}}
\newcommand{\ControlFlowTok}[1]{\textcolor[rgb]{0.13,0.29,0.53}{\textbf{#1}}}
\newcommand{\DataTypeTok}[1]{\textcolor[rgb]{0.13,0.29,0.53}{#1}}
\newcommand{\DecValTok}[1]{\textcolor[rgb]{0.00,0.00,0.81}{#1}}
\newcommand{\DocumentationTok}[1]{\textcolor[rgb]{0.56,0.35,0.01}{\textbf{\textit{#1}}}}
\newcommand{\ErrorTok}[1]{\textcolor[rgb]{0.64,0.00,0.00}{\textbf{#1}}}
\newcommand{\ExtensionTok}[1]{#1}
\newcommand{\FloatTok}[1]{\textcolor[rgb]{0.00,0.00,0.81}{#1}}
\newcommand{\FunctionTok}[1]{\textcolor[rgb]{0.13,0.29,0.53}{\textbf{#1}}}
\newcommand{\ImportTok}[1]{#1}
\newcommand{\InformationTok}[1]{\textcolor[rgb]{0.56,0.35,0.01}{\textbf{\textit{#1}}}}
\newcommand{\KeywordTok}[1]{\textcolor[rgb]{0.13,0.29,0.53}{\textbf{#1}}}
\newcommand{\NormalTok}[1]{#1}
\newcommand{\OperatorTok}[1]{\textcolor[rgb]{0.81,0.36,0.00}{\textbf{#1}}}
\newcommand{\OtherTok}[1]{\textcolor[rgb]{0.56,0.35,0.01}{#1}}
\newcommand{\PreprocessorTok}[1]{\textcolor[rgb]{0.56,0.35,0.01}{\textit{#1}}}
\newcommand{\RegionMarkerTok}[1]{#1}
\newcommand{\SpecialCharTok}[1]{\textcolor[rgb]{0.81,0.36,0.00}{\textbf{#1}}}
\newcommand{\SpecialStringTok}[1]{\textcolor[rgb]{0.31,0.60,0.02}{#1}}
\newcommand{\StringTok}[1]{\textcolor[rgb]{0.31,0.60,0.02}{#1}}
\newcommand{\VariableTok}[1]{\textcolor[rgb]{0.00,0.00,0.00}{#1}}
\newcommand{\VerbatimStringTok}[1]{\textcolor[rgb]{0.31,0.60,0.02}{#1}}
\newcommand{\WarningTok}[1]{\textcolor[rgb]{0.56,0.35,0.01}{\textbf{\textit{#1}}}}
\usepackage{graphicx}
\makeatletter
\def\maxwidth{\ifdim\Gin@nat@width>\linewidth\linewidth\else\Gin@nat@width\fi}
\def\maxheight{\ifdim\Gin@nat@height>\textheight\textheight\else\Gin@nat@height\fi}
\makeatother
% Scale images if necessary, so that they will not overflow the page
% margins by default, and it is still possible to overwrite the defaults
% using explicit options in \includegraphics[width, height, ...]{}
\setkeys{Gin}{width=\maxwidth,height=\maxheight,keepaspectratio}
% Set default figure placement to htbp
\makeatletter
\def\fps@figure{htbp}
\makeatother
\setlength{\emergencystretch}{3em} % prevent overfull lines
\providecommand{\tightlist}{%
  \setlength{\itemsep}{0pt}\setlength{\parskip}{0pt}}
\setcounter{secnumdepth}{-\maxdimen} % remove section numbering
\ifLuaTeX
  \usepackage{selnolig}  % disable illegal ligatures
\fi
\usepackage{bookmark}
\IfFileExists{xurl.sty}{\usepackage{xurl}}{} % add URL line breaks if available
\urlstyle{same}
\hypersetup{
  pdftitle={Economic analysis of femoral stem fixation method in a US Medicare population with hip fracture},
  hidelinks,
  pdfcreator={LaTeX via pandoc}}

\title{Economic analysis of femoral stem fixation method in a US
Medicare population with hip fracture}
\author{}
\date{\vspace{-2.5em}}

\begin{document}
\maketitle

The goal of this analysis is to replicate the results from Blythe et al
(2020): Fixation method for hip stems following femoral neck fracture,
but in a US-based Medicare population. The three age groups will be
similar to the original study but bounded by Medicare inclusion: 65-74,
75-84, and 85+. Data will be comprised of US-based registry data for
transition probabilities (revision, mortality) and costs where possible.
Where new data is unobtainable, we will use values from the literature.
The analysis is shown below.

\begin{Shaded}
\begin{Highlighting}[]
\FunctionTok{options}\NormalTok{(}\AttributeTok{scipen =} \DecValTok{100}\NormalTok{, }\AttributeTok{digits =} \DecValTok{5}\NormalTok{)}
\FunctionTok{library}\NormalTok{(tidyverse)}
\end{Highlighting}
\end{Shaded}

\begin{verbatim}
## -- Attaching core tidyverse packages ------------------------ tidyverse 2.0.0 --
## v dplyr     1.1.4     v readr     2.1.5
## v forcats   1.0.0     v stringr   1.5.1
## v ggplot2   3.5.1     v tibble    3.2.1
## v lubridate 1.9.3     v tidyr     1.3.1
## v purrr     1.0.2     
## -- Conflicts ------------------------------------------ tidyverse_conflicts() --
## x dplyr::filter() masks stats::filter()
## x dplyr::lag()    masks stats::lag()
## i Use the conflicted package (<http://conflicted.r-lib.org/>) to force all conflicts to become errors
\end{verbatim}

\begin{Shaded}
\begin{Highlighting}[]
\FunctionTok{library}\NormalTok{(EnvStats)}
\end{Highlighting}
\end{Shaded}

\begin{verbatim}
## 
## Attaching package: 'EnvStats'
## 
## The following objects are masked from 'package:stats':
## 
##     predict, predict.lm
\end{verbatim}

\begin{Shaded}
\begin{Highlighting}[]
\FunctionTok{library}\NormalTok{(LaplacesDemon)}
\end{Highlighting}
\end{Shaded}

\begin{verbatim}
## 
## Attaching package: 'LaplacesDemon'
## 
## The following objects are masked from 'package:EnvStats':
## 
##     dpareto, ppareto, qpareto, rpareto
## 
## The following objects are masked from 'package:lubridate':
## 
##     dst, interval
## 
## The following object is masked from 'package:purrr':
## 
##     partial
\end{verbatim}

\begin{Shaded}
\begin{Highlighting}[]
\FunctionTok{library}\NormalTok{(patchwork)}

\CommentTok{\# Model inputs}
\NormalTok{iter }\OtherTok{\textless{}{-}} \DecValTok{10000} \CommentTok{\# define number of iterations}
\NormalTok{discount }\OtherTok{\textless{}{-}} \FloatTok{0.03} \CommentTok{\# define discount rate}
\NormalTok{WTP }\OtherTok{\textless{}{-}} \DecValTok{50000} \CommentTok{\# US WTP threshold}
\FunctionTok{set.seed}\NormalTok{(}\DecValTok{888}\NormalTok{) }\CommentTok{\# repeatability}
\end{Highlighting}
\end{Shaded}

The basic structure of this population-based model is as follows: 1.
Obtain the basic prevalence of each group: HA vs THA, cemented vs
cementless, and by age, for 12 groups in total. All patients in this
model are assumed to be receiving their new hips following a femoral
neck fracture. 2. Assume that each group enters the model at time 0, or
when they first receive the prosthesis, and are assigned a cost for the
procedure. 3. Each group then proceeds through a 5 year span, iterating
by year, with the costs and utilities of stable states, revisions and
dislocations varying by year. The stable population in each year
corresponds to the surviving population who has not undergone a revision
surgery or closed reduction of dislocation. 4. The differences between
the baseline population and a default strategy of cementing or not
cementing for each age group will be calculated by assuming that all
patients receive a particular fixation type for femoral stem, rather
than using the existing prevalence of each procedure.

\begin{Shaded}
\begin{Highlighting}[]
\CommentTok{\#Call parameters and functions required to populate models}
\FunctionTok{source}\NormalTok{(}\StringTok{"./0\_Parameters\_US.R"}\NormalTok{)}
\FunctionTok{source}\NormalTok{(}\StringTok{"./1\_Model\_function.R"}\NormalTok{)}

\CommentTok{\# Run models for each age group and merge them together into one long dataframe}
\FunctionTok{source}\NormalTok{(}\StringTok{"./2\_Age\_65\_74.R"}\NormalTok{)}
\FunctionTok{source}\NormalTok{(}\StringTok{"./3\_Age\_75\_84.R"}\NormalTok{)}
\FunctionTok{source}\NormalTok{(}\StringTok{"./4\_Age\_85\_plus.R"}\NormalTok{)}

\NormalTok{models }\OtherTok{\textless{}{-}} \FunctionTok{rbind}\NormalTok{(}
\NormalTok{  model\_65\_74,}
\NormalTok{  model\_75\_84,}
\NormalTok{  model\_85\_plus}
\NormalTok{)}

\FunctionTok{str}\NormalTok{(models)}
\end{Highlighting}
\end{Shaded}

\begin{verbatim}
## tibble [120,000 x 5] (S3: tbl_df/tbl/data.frame)
##  $ Costs     : num [1:120000] -5265912 -2497948 -2423643 -2312595 -1989226 ...
##  $ QALYs     : num [1:120000] 335.7 68.6 419.9 374.2 216.7 ...
##  $ Strategy  : chr [1:120000] "All patients receive cemented fixation" "All patients receive cemented fixation" "All patients receive cemented fixation" "All patients receive cemented fixation" ...
##  $ Prosthesis: chr [1:120000] "Hemiarthroplasty" "Hemiarthroplasty" "Hemiarthroplasty" "Hemiarthroplasty" ...
##  $ Age       : chr [1:120000] "65 to 74" "65 to 74" "65 to 74" "65 to 74" ...
\end{verbatim}

\end{document}
